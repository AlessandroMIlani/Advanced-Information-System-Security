
\chapter{SSH}

\section{Introduction}



\subsection{history}
First version of SSH by Tatu Ylönen in 1995 as a response to a hacking incident (sniffer on backbone, thousands of user+pwd copied). \\ In the same year was founded the company SSH Communications Security. \\ in 1999 a OpenSSH fork appears in ObenBSD 2.6. \\
The actual version is SSH-2 by IETF (Internet Engineering Task Force) in 2006, that is:
\begin{itemize}
    \item incompatible with SSH-1
    \item improvements in security and features
    \item frequently is the only version supported nowadays
\end{itemize}

\section{Architecture}
SSH (Secure Shell) is based on a \textbf{three-layer architecture}, which includes the following protocols:

\begin{itemize}
    \item \textbf{Transport Layer Protocol:} 
    \begin{itemize}[itemsep=0pt]
        \item Establishes the initial connection.
        \item Provides server authentication.
        \item Ensures confidentiality and integrity with \textbf{Perfect Forward Secrecy (PFS)}.
        \item Supports key re-exchange (as per \textbf{RFC 4253}, it is recommended after 1 GB of data transmission or 1 hour of communication).
    \end{itemize}
    
    \item \textbf{User Authentication Protocol:} 
    \begin{itemize}
        \item Authenticates the client to the server.
    \end{itemize}
    
    \item \textbf{Connection Protocol:} 
    \begin{itemize}
        \item Supports multiple connections (channels) over a single secure channel, which is implemented by the Transport Layer Protocol.
    \end{itemize}
\end{itemize}

\subsection{Transport Layer Protocol}

\begin{figure}[ht]
    \centering
    \includegraphics[width=0.5\textwidth]{img/ssh\_TLP.png}
    \caption{SSH Transport Layer Protocol}
    \label{fig:ssh-tlp}
\end{figure}

During the TCP connection setup, the server listens on port 22 (default) and the client is the one that initiates the connection. After this, both sides must send a version string of the following form: \text{SSH-protoversion-softwareversion SP comments CR LF}, used to inidcate the capabilities of an implementation (riggers compatibility extensions). \\
\textit{All packets that follow the version string exchange is sent 
using the Binary Packet Protocol }


\subsubsection{Binary Packet Protocol}
\begin{figure}[ht]
    \centering
    \includegraphics[width=0.5\textwidth]{img/ssh\_bpp.png}
    \caption{SSH Binary Packet Protocol}
    \label{fig:ssh-bpp}
\end{figure}

The SSH Binary Packet Protocol (BPP) is structured as follows:

\textbf{Packet Length}
\begin{itemize}
    \item The packet length does \textbf{not} include the MAC or the packet length field itself.
\end{itemize}

\textbf{Payload}
\begin{itemize}[itemsep=0pt]
    \item The payload size is calculated as:
    \[
    \text{payload\_size} = \text{packet\_length} - \text{padding\_length} - 1
    \]
    \item The maximum uncompressed payload size is \textbf{32,768 bytes}.
    \item The payload might be compressed.
\end{itemize}

\textbf{Random Padding}
\begin{itemize}
    \item Padding can range from \textbf{4 to 255 bytes}.
    \item The total packet length (excluding the MAC) must be a multiple of \texttt{max(8, cipher\_block\_size)} even when a stream cipher is used.
\end{itemize}

\textbf{MAC (Message Authentication Code)}
\begin{itemize}
    \item The MAC is computed as part of the authenticate-and-encrypt schema.
    \item It is computed over the cleartext packet and an implicit sequence number.
    \item Since the MAC requires the packet to be decrypted before checking its integrity, this process could be exploited for a \textbf{denial of service (DoS)} attack.
\end{itemize}


\subsubsection{TLP key exchange}
algorithm negotiation:
\begin{itemize}[itemsep=0pt]
    \item SSH\_MSG\_KEXINIT
    \item cookie (16 random bytes)
    \item kex\_algorithms
    \item server\_host\_key\_algorithms
    \item encryption\_algorithms\_client\_to\_server, …\_server\_to\_client
    \item mac\_algorithms\_client\_to\_server, …\_server\_to\_client
    \item compression\_algorithms\_client\_to\_server, …\_server\_to\_client
    \item languages\_client\_to\_server, …\_server\_to\_client
    \item first\_kex\_packet\_follows (flag)
    \begin{itemize}
        \item attempt to guess agreed kex\_algorithm
    \end{itemize}
\end{itemize}

\subsubsection{Algorithm Specification}
\begin{itemize}[itemsep=0pt]
    \item kex\_algorithms (note: contains HASH)
    \begin{itemize}
        \item e.g. diffie-hellman-group1-sha1, ecdh-sha2-OID\_of\_curve
    \end{itemize}
    \item server\_host\_key\_algorithms
    \begin{itemize}
        \item e.g. ssh-rsa
    \end{itemize}
    \item encryption\_algorithms\_X\_to\_Y
    \begin{itemize}
        \item e.g. aes-128-cbc, aes-256-ctr, aead\_aes\_128\_gcm
    \end{itemize}
    \item mac\_algorithms\_X\_to\_Y
    \begin{itemize}
        \item e.g. hmac-sha1, hmac-sha2-256, aead\_aes\_128\_gcm
    \end{itemize}
    \item compression\_algorithms\_X\_to\_Y
    \begin{itemize}
        \item e.g. none, zlib
    \end{itemize}
    \item complete list maintained by IANA: https://www.iana.org/assignments/ssh-parameters/ssh-parameters.xhtml
\end{itemize}

\section*{SSH: Diffie-Hellman (DH) Key Agreement}

The Diffie-Hellman key agreement process in SSH follows these steps:

\begin{itemize}[itemsep=0pt]
    \item \textbf{[C] (Client)} generates a random number \( x \) and computes:
    \[
    e = g^x \mod p
    \]
    \item \textbf{[C > S]}: Client sends \( e \) to the server.
    
    \item \textbf{[S] (Server)} generates a random number \( y \) and computes:
    \[
    f = g^y \mod p
    \]
    \item \textbf{[S]} computes the shared secret:
    \[
    K = e^y \mod p = g^{xy} \mod p
    \]
    The server then computes the exchange hash:
    \[
    H = \text{HASH}( c\_version\_string \,|\, s\_version\_string \,|\, c\_kex\_init\_msg \,|\, s\_kex\_init\_msg \,|\, s\_host\_PK \,|\, e \,|\, f \,|\, K )
    \]
    \item \textbf{[S]} generates a signature \( \text{sigH} \) on \( H \) using its private key \( s\_host\_SK \). This may involve an additional hash computation on \( H \).
    
    \item \textbf{[S > C]}: Server sends its public key \( s\_host\_PK \), \( f \), and \( \text{sigH} \) to the client.
    
    \item \textbf{[C]} verifies that \( s\_host\_PK \) is indeed the server's public key.
    
    \item \textbf{[C]} computes the shared secret:
    \[
    K = f^x \mod p = g^{xy} \mod p
    \]
    The client then computes the exchange hash:
    \[
    H = \text{HASH}( \ldots )
    \]
    \item \textbf{[C]} verifies the signature \( \text{sigH} \) on \( H \).

    \item \textbf{[C, S]}: The exchange hash \( H \) becomes the session ID.
\end{itemize}

\subsection{SSH: Key Derivation}

In SSH, keys are derived using the shared secret \( K \), the exchange hash \( H \), and the \texttt{session\_id}. The derivation process is as follows:

\textbf{Initial IV (Initialization Vector)}
\begin{itemize}[itemsep=0pt]
    \item \textbf{Client to Server:} 
    \[
    \text{IV}_{\text{C to S}} = \text{HASH}( K \,||\, H \,||\, "A" \,||\, \text{session\_id} )
    \]
    \item \textbf{Server to Client:}
    \[
    \text{IV}_{\text{S to C}} = \text{HASH}( K \,||\, H \,||\, "B" \,||\, \text{session\_id} )
    \]
\end{itemize}

\textbf{Encryption Key}
\begin{itemize}[itemsep=0pt]
    \item \textbf{Client to Server:}
    \[
    \text{Key}_{\text{C to S}} = \text{HASH}( K \,||\, H \,||\, "C" \,||\, \text{session\_id} )
    \]
    \item \textbf{Server to Client:}
    \[
    \text{Key}_{\text{S to C}} = \text{HASH}( K \,||\, H \,||\, "D" \,||\, \text{session\_id} )
    \]
\end{itemize}

\textbf{Integrity Key}
\begin{itemize}[itemsep=0pt]
    \item \textbf{Client to Server:}
    \[
    \text{MAC}_{\text{C to S}} = \text{HASH}( K \,||\, H \,||\, "E" \,||\, \text{session\_id} )
    \]
    \item \textbf{Server to Client:}
    \[
    \text{MAC}_{\text{S to C}} = \text{HASH}( K \,||\, H \,||\, "F" \,||\, \text{session\_id} )
    \]
\end{itemize}

\textbf{Note:} The \texttt{session\_id} is the \( H \) value computed during the first key exchange and remains unchanged even when key re-exchange is performed.

\section{Encryption}
The used algorithms is negotiated during the key exchange phase (also key and iv are  established during in this phase), and can be diffrnt in each direction. \\
A first example of \textbf{Supported Algorithms:}
    \begin{itemize}[itemsep=0pt]
        \item \textbf{Required} (for backward compatibility):
        \begin{itemize}
            \item 3des-cbc (with three keys, i.e., 168-bit key)
        \end{itemize}[itemsep=0pt]
        \item \textbf{Recommended:}
        \begin{itemize}
            \item aes128-cbc
        \end{itemize}
        \item \textbf{Optional:}
        \begin{multicols}{3}
            \begin{itemize}[itemsep=0pt]
                \item blowfish-cbc
                \item twofish256-cbc
                \item twofish192-cbc
                \item twofish128-cbc
                \item aes256-cbc
                \item aes192-cbc
                \item serpent256-cbc
                \item serpent192-cbc
                \item serpent128-cbc
                \item arcfour
                \item idea-cbc
                \item cast128-cbc
                \item none
            \end{itemize}
        \end{multicols}
    \end{itemize}

\textit{Note:} all packets sent in one direction is a single data stream, so IV is passed from the end of one packet to the beginning of
the next one

\subsection{Encryption: Handling of IV}
In the firsty packet, the IV is randomly generated during KEX. After that, depends is it us CBC or CTR:
\begin{itemize}[itemsep=0pt]
    \item \textbf{CBC:} IV for next packet is the last encrypted block of the previous packet
    \item \textbf{CTR:} IV for next packet is the IV of the pervious packet incremeted by one
\end{itemize}


\section{MAC}
The used algorithms is negotiated during the key exchange phase and can be diffrnt in each direction. \\
\textbf{Supported Algorithms (Basic Set):}
\begin{multicols}{2}
        \begin{itemize}[itemsep=0pt]
            \item \textbf{hmac-sha1 (required)} [key length = 160-bit] (backward compatibility)
            \item \textbf{hmac-sha1-96 (recommended)} [key length = 160-bit]
            \item \textbf{hmac-md5 (optional)} [key length = 128-bit]
            \item \textbf{hmac-md5-96 (optional)} [key length = 128-bit]
        \end{itemize}
\end{multicols}

Even if these are the value inidicate in the RFC, for security we always need to use algorithms in the \textbf{SHA-2 family}. \\
The MAC is computed in the following way (\textbf{ MAC = mac( key, seq\_number \,|\, cleartext\_packet )}):
\begin{multicols}{2}
    \begin{enumerate}[itemsep=0pt]
        \item sequence number is implicit, not sent with the packet
        \item sequence number is represented on 4 bytes
        \item seq\_number initially 0 and incremented after each packet
        \item seq\_number never reset (even if keys/algos are renegotiated)
    \end{enumerate}
\end{multicols}

\section{SSH: Peer Authentication}

\subsection{Server}
The server authetication is done by an symmetric challenge-response (explicit server signature of the key exchange hash H). \\
So the client need to locally store the server public keys (not good). Typically are saved in \texttt{\textasciitilde/.ssh/known\_hosts} and, is a key is absenty then it's offered at first connection with a TOFU (Trust On First Use) policy, \textbf{Very Danger}. \\
For good practice we need to protect \texttt{known\_hosts} for \textbf{authentication and integrity} and \textbf{periodic audit/review} all the \texttt{known\_hosts} files to quickly detect added/deleted hosts or changed keys

\subsection{Client}
Can be done in two ways:
\begin{itemize}[itemsep=0pt]
    \item \textbf{Pusername and password:} the client sends the password to the server only after the protected channel is created. Usefull for avoid sniffing, but open to other attacks (e.g. on-line password enumeration)
    \item \textbf{asymmetric challenge-response:} server locally stores the public keys of the users allowed to connect as a local user (typically in \texttt{\textasciitilde/.ssh/authorized\_keys}).
\end{itemize}
Some good practices are the protection of authorized\_keys for authentication and integrity and the periodic audit/review of all the authorized\_keys files to quickly detect added/deleted users or changed keys.